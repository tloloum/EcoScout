
Chaque fonctionnalité clé sera représentée par une page dédiée dans l'application pour assurer une navigation intuitive. Voici comment les différentes fonctionnalités seront intégrées en tant que pages de l'application :

\begin{itemize}[label=-]
    \item \textbf{Page d'Accueil (Dashboard) :} Cette page constituera le point central de l'application, permettant aux utilisateurs de gérer leur compte, afficher un résumé personnalisé de leurs activités, et accéder rapidement aux fonctionnalités principales.

    \item \textbf{Page des Utilisateurs (Profil) :} Les utilisateurs auront une page dédiée pour créer et gérer leurs comptes. Ils pourront également consulter leurs informations personnelles et préférences liées à l'application.

    \item \textbf{Page des Statistiques (Calcul d'Impact) :} Les statistiques globales concernant les événements, les structures, et les individus seront disponibles sur cette page. Les moyennes, les comparaisons sur le temps, et d'autres indicateurs écologiques seront présentés de manière claire. Cette page offrira aux utilisateurs la possibilité de visualiser l'impact écologique de leurs actions, de leurs dépenses carbones, avec une distinction entre les dépenses communes et individuelles.. Le calcul d'impact sera alimenté par la base de données de l'ADEME. 

    \item \textbf{Page des Structures :} La création et la gestion de structures seront accessibles depuis cette page. Les utilisateurs pourront inviter d'autres membres, définir des relations entre eux, et créer des structures de différentes tailles.

    \item \textbf{Page des Badges :} Bien que secondaire, la page des badges présentera le côté ludique de l'application, récompensant les utilisateurs pour leurs actions écologiques.

    \item \textbf{Page d'Organisation d'Événements :} Les utilisateurs pourront organiser des événements, tels que des week-ends de groupe, directement depuis cette page.

\end{itemize}
Nous allons maintenant voir le parcours que pourra faire un utilisateur sur l'application. \\
\newline
\textbf{Etape 1 - Connexion - Page d'accueil}\\
Sur la page d'accueil, un \textit{utilisateur} peut se connecter via son \textit{login}. Si ce dernier n'appartient pas encore à la base de données, il peut rentrer les différentes informations (Nom / Prenom / Adresse mail / Login) pour pouvoir créer son compte. \\
\newline 
La protection de la connexion sera réalisée avec un mot de passe ou alors un système de code à rentrer envoyé à l'utilisateur. \\
\newline
\textbf{Etape 2 - Créer une structure - Page des structures}\\
Une fois connecté, l'utilisateur souhaite créer son équipe \textit{L’Unité Louveteaux-Jeannettes} et rejoindre via cette équipe le \textit{Groupe de Chamarande}.\\
\newline
La première étape est donc de créer \textit{L’Unité Louveteaux-Jeannettes} et de la partager aux membres. L'utilisateur se rend donc dans le menu structures et \textit{Créer une structure}. Il va devoir rentrer le nom de la structure, une description et, de façon optionnelle la date qui représente la date de création de la structure. Si cette date n'est pas spécifiée, la date de création sera automatiquement définie comme la date à laquelle la structure a été créée dans l'application. L'utilisateur qui a crée cette structure est considéré comme un responsable de cette structure.\\
\newline
Une fois la structure créé, elle devrait être visible dans la liste des structures. Si on la sélectionne on peut voir les informations de la structure comme le nombre de membres par exemple. \textit{L’Unité Louveteaux-Jeannettes} ne possède ici qu'un seul membre qui est l'utilisateur qui a créé cette structure. Il peut, car il est un responsable de la structure, inviter des nouveaux membres, trouvables via login ou mail. Si d'autres membres rejoignent la structure, les utilisateurs qui sont responsables de la structure peuvent nommer d'autres responsables en plus.\\
\newline
\textbf{Etape 3 - Rejoindre une structure - Page des structures}\\
Deux situations sont possibles pour rejoindre une structure. 
\begin{itemize}
    \item Si on a reçu une invitation (par notification ou par mail), on peut accepter l'invitation et rejoindre la structure (même situation pour un événement). Si la personne qui souhaite rejoindre n'a pas encore de compte (donc forcément dans une situation où l'invitation a été reçue par l'autre). 
    \item Si un utilisateur décide de rejoindre une structure directement. Pour réaliser cela, il suffit de se rendre sur la page de la structure et de cliquer sur rejoindre. A partir de ce moment-là, un menu lui demandera s'il souhaite rejoindre en tant que personne ou rejoindre via une structure dont il est responsable (il sera donc possible d'en sélectionner un). Il est donc possible d'appartenir directement ou indirectement à une structure. L'invitation devra donc être accepté par un responsable.
\end{itemize}
\textbf{Remarque :} Dans le cas où l'on souhaite donner les droits de responsables à une "sous-structure" qui appartient à notre structure, alors seulement les responsables de cette "sous-structure" seront responsable de la structure.\\
\newline 
\textbf{Etape 4 - Evènement}\\
De la même manière que pour les structures, il est possible de créer et de rejoindre un événement. Un événement est défini par une date de début, une date de fin, une description et, si l'on souhaite, des défis. \\
\newline 
\textbf{Etape 5 - Impact}\\
Il est possible, pour un événement de rajouter des actions qui ont un impact écologique. Lorsqu'un utilisateur ajoute une action, il l'ajoute dans un événement. Cependant, cette action est reliée à l'utilisateur, elle sera donc visible dans cet événement, mais aussi sur le profil de l'utilisateur et dans les structures où l'utilisateur est présent si ses structures appartiennent à cet événement. 



