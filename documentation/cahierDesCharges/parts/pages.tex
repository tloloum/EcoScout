
Chaque fonctionnalité clé sera représentée par une page dédiée dans l'application pour assurer une navigation intuitive. Voici comment les différentes fonctionnalités seront intégrées en tant que pages de l'application :

\begin{itemize}[label=-]
    \item \textbf{Page d'Accueil (Dashboard) :} Cette page constituera le point central de l'application, permettant aux utilisateurs de gérer leur compte, afficher un résumé personnalisé de leurs activités, et accéder rapidement aux fonctionnalités principales.

    \item \textbf{Page des Utilisateurs (Profil) :} Les utilisateurs auront une page dédiée pour créer et gérer leurs comptes. Ils pourront également consulter leurs informations personnelles et préférences liées à l'application.

    \item \textbf{Page des Statistiques (Calcul d'Impact) :} Les statistiques globales concernant les événements, les groupes, et les individus seront disponibles sur cette page. Les moyennes, les comparaisons sur le temps, et d'autres indicateurs écologiques seront présentés de manière claire. Cette page offrira aux utilisateurs la possibilité de visualiser l'impact écologique de leurs actions, de leurs dépenses carbones, avec une distinction entre les dépenses communes et individuelles.. Le calcul d'impact sera alimenté par la base de données de l'ADEM. 

    \item \textbf{Page des Groupes :} La création et la gestion de groupes seront accessibles depuis cette page. Les utilisateurs pourront inviter d'autres membres, définir des relations entre eux, et créer des groupes de différentes tailles.

    \item \textbf{Page des Badges :} Bien que secondaire, la page des badges présentera le côté ludique de l'application, récompensant les utilisateurs pour leurs actions écologiques.

    \item \textbf{Page d'Organisation d'Événements :} Les utilisateurs pourront organiser des événements, tels que des week-ends de groupe, directement depuis cette page.

\end{itemize}

