\subsection{Spécifications Indispensables}

\begin{itemize}
    \item \textbf{Créer un compte :} Définir un utilisateur avec ses informations primordiales (login, nom, prénom, mail, numéro d'adhérent)
    \item \textbf{Créer une structure :} Capacité d'un utilisateur de créer une structure. Celui-ci devient responsable et aura alors les droits de modifications de la structure lui permettant de : 
    \begin{itemize}
        \item Rajouter/Supprimer des informations à renseigner pour rejoindre cette structure
        \item Créer des utilisateurs dans cette structure avec des informations de bases(en attendant que les personnes rejoignent la structure avec leurs propres comptes).
        \item Faire rejoindre sa structure à une autre (hiérarchie des structures)
        \item Inviter un utilisateur à rejoindre sa structure via un lien d'invitation. 
    \end{itemize}
    \item \textbf{Ajout d'une action ayant un impact écologique} : Capacité d'un utilisateur/d'une structure de rajouter un impact écologique (exemple : un trajet en voiture)
\end{itemize}

\subsection{Spécifications Importantes}

Étant donné que le terme 'important' peut prêter à confusion en raison de sa nature vague, les spécifications qualifiées d'importantes sont clairement hiérarchisées et triées par ordre de priorité pour garantir une compréhension et une mise en œuvre précises de ces éléments clés du projet

\begin{itemize}
    \item \textbf{1. Organiser un événement :} Un utilisateur ou le responsable d'une structure à la capacité de définir un événement sur une durée fixe et de rajouter/inviter des utilisateurs/structures dans cet événement. Un événement peut contenir plusieurs événements (ex : une année est considérée comme un événement durant laquelle les week-ends sont aussi des événements).
    \item \textbf{2. Suppression d'une structure ou d'un événement :} Capacité d'un responsable à supprimer un événement ou une structure crée.
    \item \textbf{3. Partage des droits sur une structure :} Un responsable peut partager les droits de modification d'une structure avec d'autres utilisateurs
    \item \textbf{4. Affichage de statistiques globales :} Calcul et affichage de statistiques sur son impact écologique d'une structure ou d'un événement (moyenne, progression/comparaison des impacts sur une durée). 
\end{itemize}

\subsection{Spécifications Secondaires}
\begin{itemize}
    \item \textbf{Ajout d'objectifs :} Un responsable peut rajouter des objectifs pour sa structure ou son événement.
    \item \textbf{Ajout défis :} Un responsable peut regrouper un ensemble d'objectifs à atteindre sur un temps donnée (par exemple lors d'un événement). 
    \item \textbf{Attribution de récompense à la réalisation de défis :} L'accomplissement de défis peut amener à la récupération de récompense, comme un badge, attribué à un utilisateur. 
\end{itemize}

% \begin{table}[h]
% \centering
% \begin{tabularx}{\textwidth}{|X|X|X|X|X|}
%     \hline
%     \textbf{Fonctionnalité} & \textbf{Description} & \textbf{Complexité} & \textbf{Priorité} & \textbf{Justification} \\
%     \hline
%     \textbf{Créer un compte} & Définir un utilisateur (nom, prénom, ... ) & Simple & \textit{Indispensable} & Création de BDD \\
%     \hline
%     \textbf{Faire des groupes} & Rejoindre plusieurs utilisateurs, avec un/des administrateur(s) qui a le droit de modification (suppression de données non valables) & Simple & \textit{Indispensable} & BDD \\
%     \hline
%     \textbf{Organisation d'événements} & Définir un événement sur une durée fixe et rajouter des utilisateurs/groupes dans cet événement & Moyen & \textit{Important} & \\
%     \hline
%     \textbf{Lien d'invitation dans un groupe} & Permettre à un administrateur d'inviter des utilisateurs dans un groupe & Moyen & \textit{Indispensable} & \\
%     \hline
%     \textbf{Définir un événement public / privé} & Permettre à des personnes de s'ajouter sans invitation à un événement/groupe & Moyen & \textit{Secondaire, à proposer} & \\
%     \hline
%     \textbf{Calcul de l'impact} & Calculer l'impact écologique d'un groupe/d'une personne/d'un événement. Sur un temps choisi/une action particulière (ex: le trajet) & Difficile & \textit{Important} & \\
%     \hline
%     \textbf{Ajout d'un impact écologique} & Un utilisateur/un groupe peut rajouter des actions qui ont un impact écologique & Moyen & \textit{Indispensable} & \\
%     \hline
% \end{tabularx}
% \end{table}

% \newpage 

% \begin{table}[h]
%     \centering
%     \begin{tabularx}{\textwidth}{|X|X|X|X|X|}
%     \hline
%     \textbf{Rechercher une action qui a un impact écologique} & Rechercher une action pour la rajouter dans les impacts d'un utilisateur / groupe / événements & Simple/Moyen & \textit{Important} & Utilisation de la BDD de l'ADEM \\
%     \hline
%     \textbf{Côté ludique (badge,...)} & Rajout d'un côté ludique à l'application, et ne pas donner un aspect de compétition & Difficile & \textit{Secondaire} & Ensemble de fonctionnalités secondaires, complexité difficile à calculer \\
%     \hline
%     \textbf{Statistiques globales (progression sur le temps, impacts des actions)} & Calcul des statistiques globales d'un groupe/événements. (moyennes, comparaison sur le temps,...) & Moyen & \textit{Important} & \\
%     \hline
%     \textbf{Suppression d'un groupe / événements} & Capacité de pouvoir supprimer un groupe/événements par un administrateur de celui-ci & Facile & \textit{Important} & \\
%     \hline
%     \textbf{Suppression automatique (événements / groupes trop anciens / passés)} & Suppression automatique des événements ou des groupes dont l'activité est inexistante depuis trop longtemps & Moyen & \textit{Secondaire} & \\
%     \hline
%     \textbf{Ajout de droits sur un groupe} & L'administrateur initial d'un groupe ou d'un événement peut rajouter des droits à certains utilisateurs sur ce groupe & Facile &  & \\
%     \hline
% \end{tabularx}
% \end{table}

\newpage 
